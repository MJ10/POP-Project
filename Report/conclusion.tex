\chapter{Conclusion}

\section{Results}
\paragraph{}
At the end, we were able to compile a \textit{.jar} file which can be run on any system with JRE available. In the project we were able to successfully implement the basic game of Chess with a GUI, using Java Swing Library. The game utilized several key concepts of object oriented programming. This includes Inheritance, Polymorphism(Runtime), Data Abstraction and Data Encapsulation. The project also utilizes features of the Java programming language to implement these concepts, such as Interfaces.

\section{Conclusion}
\paragraph{}
In the project we were able to apply the concepts of Object Oriented Programming to the implementation of the game of Chess in Java, for two players. We were also able to utilize the Java Swing GUI library which enabled us to build the Graphical User Interface for the game. The following are few of the concepts that were used among others:

\begin{itemize}
    \item Inheritance - Interfaces, Marker Interfaces, Hierarchial Inheritance
    \item Runtime Polymorphism - Method Overriding
    \item Data Abstraction - Abstract Classes and Methods
    \item Data Encapsulation - Access Modifiers
    \item GUI Design - Swing 
\end{itemize}

\section{References}
\paragraph{}
[1] Java Swing

\href{https://docs.oracle.com/javase/tutorial/uiswing/start/index.html}{https://docs.oracle.com/javase/tutorial/uiswing/start/}

\paragraph{}
[2] Java Swing Tutorials

\href{https://www.tutorialspoint.com/swing/}{https://www.tutorialspoint.com/swing/}

\paragraph{}
[3] Images of Chess Pieces.

\href{https://commons.wikimedia.org/wiki/Category:PNG_chess_pieces/Standard_transparent}{https://commons.wikimedia.org/wiki/Category:PNG\_chess\_pieces/}

\section{Appendix}

\paragraph{}
Some important snippets have been attached at the end of the report.

\paragraph{}
The full code is available at \href{https://github.com/MJ10/POP-Project}{https://github.com/MJ10/POP-Project}
